\documentclass{book}[oneside]
\usepackage{CJKutf8}
\usepackage{amsmath}
\usepackage{amsfonts}
\usepackage{amsthm}
\usepackage{titlesec}
\usepackage{titletoc}
\usepackage{xCJKnumb}
\usepackage{tikz}
\titleformat{\chapter}{\centering\Huge\bfseries}{第\, \xCJKnumber{\thechapter}\,
    章}{1em}{}
  % \renewcommand{\chaptermark}[1]{\markboth{第 \thechapter 章}{}}
\usepackage{mathrsfs}

\newtheorem{Def}{定义}[chapter]
\newtheorem{Thm}{定理}[chapter]
\newtheorem{Exercise}{练习}[chapter]

\newtheorem*{Example}{例}


\begin{document}
\begin{CJK*}{UTF8}{gbsn}
  \title{离散数学讲义}
  \author{陈建文}
  \maketitle
  % \tableofcontents
  
  \underline{课程学习目标:}
\begin{enumerate}
\item 训练自己的逻辑思维能力和抽象思维能力
\item 训练自己利用数学语言准确描述计算机科学问题的能力
\end{enumerate}

\underline{学习方法:}
\begin{enumerate}
\item MOOC自学
\item 阅读该讲义
\item 做习题
\item 学习过程中有不懂的问题,在课程QQ群中与老师交流
\end{enumerate}

\underline{授课教师QQ:}2129002650

% \underline{老师的话:}

% 我们首先要明确本门课程的两个学习目标。我们学习的过
% 程是围绕两个学习目标进行的,我们课程考试的内容也是按照我们的学习目标进
% 行设置的。

% 1)认真理解本课程中的每一个数学概念,对于学好本门课程是至关重要
% 的。这些数学概念将陪伴我们整个学习计算机科学的生涯。阅读并理解这些数学
% 概念的过程,对于培养我们将来描述计算机科学问题时准确的给出数学概念的能
% 力是非常重要的,这对应于我们抽象思维能力的培养。

% 2)认真理解本课程中的每一个证明过程,这对于培养我们的逻辑思维能力是至
% 关重要的。要想通过本课程的学习提升自己的逻辑思维能力,认真阅读并理解一
% 些重要定理的证明过程是必不可少的一个关键环节。如果我们只是记住一些定理,
% 没有理解定理的证明过程,要想取得一个好的考试成绩是很困难的。举例来说,
% 在中学时,我们学习了圆面积的计算公式$S=\pi
% r^2$,只需要掌握当$r=2$时,能够计算出$S=4\pi$即可。我们做了大量这样的习
% 题,最后能够取得一个好的考试成绩。但如果考试时要求我们证明$S=\pi r^2$,
% 如果我们没有理解该公式的证明过程,还能取得一个好的考试成绩吗?我们的期
% 末考试将以证明题为主,这与我们的学习目标是一致的。


  \setcounter{chapter}{0}
  \chapter{集合}
\begin{Ex}[课本第8页第3题]
  \mbox{} \par \noindent
  
写出方程
\begin{equation*}
x^2+2x+1=0
\end{equation*}
的根构成的集合。
\end{Ex}
\begin{proof}[证明]
$\{-1\}$
\end{proof}

\end{CJK*}
\end{document}





%%% Local Variables:
%%% mode: latex
%%% TeX-master: t
%%% End:



x