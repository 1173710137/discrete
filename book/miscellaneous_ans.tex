\chapter{综合题}

\begin{Ex}
  珍珠四颗,有真有假,不能用眼鉴别。真珍珠重量相同且为p,假珍珠重量也相同且为q,
  $p > q$。用秤(不是天平)仅称三次,称出真假,应该怎样做?
\end{Ex}
\begin{proof}[解]
  设四颗珍珠分别为$p_1$,$p_2$,$p_3$,$p_4$,其重量分别为$x_1$,$x_2$,$x_3$,
  $x_4$。第一次将$p_1$和$p_2$放在一起称,设得到的重量为$a$;第二次将$p_1$和$p_3$
  放在一起称,设得到的重量为$b$;第三次将$p_2$,$p_3$和$p_4$放在一起称,设得到的
  重量为$c$。
  于是可以得到
  \begin{equation}
    \begin{cases}
      x_1 + x_2 = a\\
      x_1 + x_3 = b\\
      x_2 + x_3 + x_4 = c
    \end{cases}
  \end{equation}
  令$y_1=\frac{x_1-q}{p-q}$,$y_2=\frac{x_2-q}{p-q}$,$y_3=\frac{x_3-q}{p-q}$,
  $y_4=\frac{x_4-q}{p-q}$,
  可以得到
  \begin{equation}\label{e1}
    \begin{cases}
      y_1 + y_2 = \frac{a-2q}{p-q}\\
      y_1 + y_3 = \frac{b-2q}{p-q}\\
      y_2 + y_3 + y_4 = \frac{c-3q}{p-q}
    \end{cases}
  \end{equation}
  以上三个式子相加,可得
  \begin{equation}
    2(y_1 + y_2 + y_3) + y_4 = \frac{a-2q}{p-q} + \frac{b-2q}{p-q} + \frac{c-3q}{p-q}
  \end{equation}
  根据上式右端为偶数或奇数,可得$y_4$为$0$或1。带入方程组\eqref{e1}可得$y_1$,
  $y_2$,$y_3$的值为$0$或$1$,从而相应的可以判断$x_1$,$x_2$,$x_3$,$x_4$的值为
  $p$或$q$。
\end{proof}