\chapter{集合}
\begin{Ex}
设$A$,$B$,$C$是集合,证明$(A\bigtriangleup B)\bigtriangleup C =
A\bigtriangleup (B\bigtriangleup C)$。
\end{Ex}
\begin{proof}[证明]
  
  因为
  \begin{equation}
  x \in A \bigtriangleup B \Leftrightarrow
  (x \in A \land x \notin B) \lor (x \notin A \land x
  \in B),    
  \end{equation}

  所以
  \begin{equation}
    \begin{split}
      x \notin A \bigtriangleup B &\Leftrightarrow
  (x \notin A \lor x \in B) \land (x \in A \lor x
  \notin B)\\
  &\Leftrightarrow (x \notin A \land x \notin B) \lor (x \in A \land x \in B )
    \end{split}
  \end{equation}

  于是
  \begin{equation}\label{xor1}
    \begin{split}
      &x \in (A \bigtriangleup B) \bigtriangleup C\\
      &\Leftrightarrow (x \in A \bigtriangleup B \land x \notin C) \lor (x \notin A \bigtriangleup B \land x \in C)\\
      &\Leftrightarrow (((x \in A \land x \notin B) \lor (x \notin A \land x \in B)) \land x \notin C)\\
      &\lor (((x \notin A \land x \notin B) \lor (x \in A \land x \in B )) \land x \in C)\\
      &\Leftrightarrow (x \in A \land x \notin B \land x \notin C) \lor (x \notin A \land x \in B \land x \notin C)\\
      &\lor (x \notin A \land x \notin B \land x \in C) \lor (x \in A \land x \in B \land x \notin C)
    \end{split}
  \end{equation}

  \begin{equation}\label{xor2}
    \begin{split}
      &x \in A \bigtriangleup (B \bigtriangleup C)\\
      &\Leftrightarrow x \in (B \bigtriangleup C) \bigtriangleup A\\
      &\Leftrightarrow (x \in A \land x \notin B \land x \notin C) \lor (x \notin A \land x \in B \land x \notin C)\\
      &\lor (x \notin A \land x \notin B \land x \in C) \lor (x \in A \land x \in B \land x \notin C)
    \end{split}
  \end{equation}
  其中\eqref{xor2}式的第二行由对称差运算的交换律得到,\eqref{xor2}式的第三行由与
   \eqref{xor1}式的对称性得到。

  由\eqref{xor1}式和\eqref{xor2}式可得
  $(A\bigtriangleup B)\bigtriangleup C = A\bigtriangleup (B\bigtriangleup C)$。
\end{proof}
 