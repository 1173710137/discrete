\documentclass{beamer}
%\usepackage{beamerthemesplit}
\usepackage{CJKutf8}
\usepackage{tikz}
\setbeamertemplate{theorems}[numbered]

\usepackage{ragged2e}
%\justifying\let\raggedright\justifying


\begin{document}
\begin{CJK*}{UTF8}{gbsn}

\newtheorem{Thm}{定理}[section]
\newtheorem{Cor}{推论}[section]
\theoremstyle{definition}
\newtheorem{Def}{定义}[section]
\theoremstyle{example}
\newtheorem*{Ex}{例:}
\newtheorem{Exercise}{习题}

\date{}
\author{陈建文}

\title{第二章 映射}
\begin{frame}
  \titlepage
\end{frame}  
\section{映射}
\begin{frame}
  \frametitle{1. 映射}
  
  % \begin{Def}
  %   设$X$和$Y$是两个非空集合,$P(x,y)$是定义在$X\times Y$上的一个二元谓词,且满足以下两个条件:
  %   \begin{enumerate}[(1)]
  %   \item 对$X$的每一个元素$x$, 存在$y\in Y$, 使得$P(x,y)$取值为真;
  %   \item 如果$P(x,y_1)$,$P(x, y_2)$同时取值为真,则$y_1 = y_2$。
  %   \end{enumerate}
  %   则$P(x,y)$确定了一个\alert{映射}$f:X\to Y$, 对任何的$x\in X$,$f(x)$定义为$Y$中使得$P(x,y)$为真的唯一的$y$。这里$y$称为$x$在$f$下的\alert{象},而$x$称为$y$的\alert{原象}。
  %   $X$称为$f$的\alert{定义域};集合$\{f(x) | x \in X\}$称为$f$的\alert{值域},记为$Im(f)$。
  % \end{Def}
  \begin{Def}
    设$X$和$Y$为两个非空集合。一个从$X$到$Y$的\alert{映射}$f$为一个法则,根据$f$,对$X$中的每个元素$x$都有$Y$中唯一确定的元素$y$与之对应。
    从$X$到$Y$的映射$f$常记为$f:X\to Y$。
  \end{Def}\pause

  \begin{Def}
    设$X$和$Y$为两个非空集合。一个从$X$到$Y$的映射为一个满足以下两个条件的$X\times Y$的子集$f$:
    \begin{enumerate}
    \item 对$X$的每一个元素$x$,存在一个$y\in Y$,使得$(x,y) \in f$;
    \item 若$(x,y)\in f$,$(x,y')\in f$,则$y=y'$。
    \end{enumerate}
    $(x,y)\in f$记为$y=f(x)$。
  \end{Def}\pause

  \begin{Def}\justifying\let\raggedright\justifying
    设$f$为从集合$X$到集合$Y$的映射,$f:X\to Y$, 如果$y = f(x)$,$y$称为$x$在$f$下的\alert{象},而$x$称为$y$的\alert{原象}。$X$称为$f$的\alert{定义域};集合$\{f(x) | x \in X\}$称为$f$的\alert{值域},记为$Im(f)$。
  \end{Def}
\end{frame}
\begin{frame}
  \frametitle{1. 映射}

  \begin{Def}
    设$f:X\to Y$,$A\subseteq X$,当把$f$的定义域限制在$A$上时,就得到了一个
    $\phi: A\to Y$,$\forall x \in A$,$\phi(x) = f(x)$。$\phi$称为$f$在$A$上的
    \alert{限制},并且常用$f|A$来表示$\phi$。反过来,我们也称$f$为$\phi$在$X$上的\alert{扩张}。
  \end{Def}\pause
    \begin{Def}
    设$f:A \to Y$,$A \subseteq X$, 则称$f$为$X$上的一个\alert{部分映射}。
  \end{Def}\pause
  \begin{Def}
    两个映射$f$与$g$称为是\alert{相等}的当且仅当$f$和$g$都为$X$到$Y$的映射,并且$\forall x \in X$总有$f(x) = g(x)$。
  \end{Def}\pause
  \begin{Def}
    设$f:X\to Y$,如果$\forall x \in X, f(x) = x$,则称$f$为$X$上的恒等映射。$X$上的恒等映射常记为$I_X$。
  \end{Def}

\end{frame}

\begin{frame}
  \frametitle{1.映射}
  \begin{Def}\justifying\let\raggedright\justifying
    设$f:X\to Y$,如果$\forall x_1, x_2 \in X$, 只要$x_1 \neq x_2$,  就 有 $f(x_1) \neq f(x_2)$,   则称$f$为从$X$到$Y$的\alert{单射}。
  \end{Def}
  \begin{Def}\justifying\let\raggedright\justifying
    设$f:X\to Y$, 如果$\forall y \in Y$, $\exists x \in X$使得 $f(x) = y$, 则称$f$为从$X$到$Y$的\alert{满射}。
  \end{Def}
  \begin{Def}\justifying\let\raggedright\justifying
    设$f:X\to Y$,如果$f$既是单射又是满射,则称$f$为从$X$到$Y$的\alert{双射},或者称$f$为从$X$到$Y$的\alert{一一对应}。
  \end{Def}

\end{frame}
\section{鸽笼原理}
\begin{frame}
  \frametitle{2. 鸽笼原理}
  \begin{Thm}[鸽笼原理]
    如果把$n+1$个物体放到$n$个盒子里,则必有一个盒子里至少放了两个物体。
  \end{Thm}
  \pause
  \begin{Ex}
    已知$m$个整数$a_1,a_2,\ldots,a_m$,试证:存在两个整数 $k$, $l$, \\ $0\leq k < l \leq m$,使得$a_{k+1}+a_{k+2}+\ldots+a_{l}$能被$m$整除。
  \end{Ex}
\end{frame}

\begin{frame}
  \frametitle{2. 鸽笼原理}
  \begin{Thm}[鸽笼原理的强形式]
    设$q_1$,$q_2$,$\ldots$,$q_n$为$n$个正整数。如果把 $q_1 + q_2 + \cdots + q_n$  $- n$ $ + 1$ 个物体放到$n$个盒子中,则或者第一个盒子中至少含有$q_1$个物体,或者第二个盒子中至少含有$q_2$个物体,$\ldots$,
    或者第$n$个盒子中至少含有$q_n$个物体。
  \end{Thm}\pause
  \begin{Cor}
    如果把$n(r-1) + 1$个物体放入$n$个盒子里,则至少有一个盒子里放了不少于$r$个物体。
  \end{Cor}\pause
  \begin{Cor}
    如果把$n$个正整数$m_1, m_2, \ldots, m_n$的平均值\[\frac{m_1 + m_2 + \ldots + m_n}{n} > r - 1,\] 则$m_1, m_2, \ldots, m_n$中至少有一个正整数不小于$r$。
  \end{Cor}
\end{frame}
\section{映射的一般性质}
\begin{frame}
  \frametitle{3. 映射的一般性质}
  \begin{Def}
    设$f:X\to Y$,$A \subseteq X$,$A$在$f$下的\alert{象}定义为\[f(A)=\{f(x)|x\in A\}\]
  \end{Def}\pause
  \begin{Ex}
    设$f:\{-1,0,1\}\to \{-1,0,1\}$,$f(x)=x^2$,$f(\{-1,0\})$=?
  \end{Ex}
\end{frame}
\begin{frame}
  \frametitle{3. 映射的一般性质}
  \begin{Def}
    设$f:X\to Y$,$B \subseteq Y$,$B$在$f$下的\alert{原象}定义为\[f^{-1}(B)=\{x\in X|f(x)\in B\}\]
  \end{Def}\pause
  \begin{Ex}
    设$f:\{-1,0,1\}\to \{-1,0,1\}$,$f(x)=x^2$,$f^{-1}(\{-1,0\})$=?
  \end{Ex}
\end{frame}

\begin{frame}
  \frametitle{3. 映射的一般性质}
  \begin{Thm}
    设$f:X\to Y$,$A \subseteq Y$,$B \subseteq Y$, 则
    \begin{enumerate}[(1)]
    \item $f^{-1}(A \cup B) = f^{-1}(A) \cup f^{-1}(B)$
    \item $f^{-1}(A \cap B) = f^{-1}(A) \cap f^{-1}(B)$
    \item $f^{-1}(A^c)=(f^{-1}(A))^c$
    \item $f^{-1}(A \bigtriangleup B) = f^{-1}(A) \bigtriangleup f^{-1}(B)$
    \end{enumerate}
  \end{Thm}
    \begin{Thm}
    设$f:X\to Y$,$A \subseteq X$,$B \subseteq X$, 则
    \begin{enumerate}[(1)]
    \item $f(A \cup B) = f(A) \cup f(B)$
    \item $f(A \cap B) \subseteq f(A) \cap f(B)$
    \item $f(A \bigtriangleup B) \supseteq f(A) \bigtriangleup f(B)$
    \end{enumerate}
  \end{Thm}

\end{frame}

\section{映射的合成}
\begin{frame}
  \frametitle{4. 映射的合成}
  \begin{Def}
    设$f:X\to Y$,$g:Y\to Z$为映射,映射$f$与$g$的\alert{合成} $g\circ f:X\to Z$定义为\[(g\circ f)(x) = g(f(x))\]
  \end{Def}
  \pause
\begin{Thm}
  设$f:X \to Y$,$g:Y\to Z$,$h:Z\to W$ 为映射,则 \[ (h \circ g) \circ f = h \circ (g \circ f) \]
\end{Thm}

\end{frame}

\section{逆映射}
\begin{frame}
  \frametitle{5. 逆映射}
  \begin{Def}
     设$f:X\to Y$为双射,$f$的\alert{逆映射}$f^{-1}:Y\to X$定义为:对任意的$y\in Y$,存在唯一的$x$使得$f(x)=y$,则$f^{-1}(y)=x$。
   \end{Def}
   \pause
     \begin{block}{定义5.1'}
     设$f:X\to Y$为一个映射。如果存在一个映射$g:Y\to X$使得\[f\circ g = I_{Y} \text{且} g\circ f = I_{X},\]则称映射$f$是可逆的,而$g$称为$f$的\alert{逆映射}。
  \end{block}
\end{frame}
\begin{frame}
  \frametitle{5. 逆映射}
  \begin{Thm}
    设$f:X\to Y$为可逆映射,则$(f^{-1})^{-1}=f$。
  \end{Thm}
  \pause
  \begin{Thm}
    设$f:X\to Y$,$g:Y\to Z$都为可逆映射,则$g\circ f$也为可逆映射并且$(g\circ f)^{-1} = f^{-1}\circ g^{-1}$。
  \end{Thm}
\end{frame}
\begin{frame}
  \frametitle{5. 逆映射}
  \begin{Def}\justifying\let\raggedright\justifying
    设$f:X\to Y$为一个映射,如果存在一个映射 $g:Y\to X$ 使 得 $g\circ f = I_X$,
    则称$f$为\alert{左可逆}的,$g$称为$f$的\alert{左逆映射};如果存在一个映射
    $h:Y\to X$ 使 得 $f\circ h=I_Y$,则称$f$为\alert{右可逆}的,$h$称为$f$的\alert{右逆映射}。
  \end{Def}\pause
  \begin{Thm}
    设$f:X\to Y$为一个映射,则
    \begin{enumerate}
    \item $f$左可逆当且仅当$f$为单射;
    \item $f$右可逆当且仅当$f$为满射。
    \end{enumerate}
  \end{Thm}
\end{frame}
\section{置换}
\begin{frame}
  \frametitle{6. 置换}
  \begin{Def}
    有限集合$S$到自身的一一对应称为$S$上的一个\alert{置换}。如果$|S| = n$, 则$S$上的置换就说成是\alert{$n$次置换}。
  \end{Def}\pause
  若$\alpha$与$\beta$是两个$n$次置换,当把$\beta$的表示式中的上一行按$\alpha$的下一行的顺序写出时,则$\alpha \beta$的下一行就是$\beta$的新表示式中的下一行。
 \end{frame}
 \begin{frame}
   \frametitle{6. 置换}
   \begin{Def}
     设$\sigma$是$S$上的一个$n$次置换,若$i_1\sigma=i_2$,$i_2\sigma = i_3$, $\cdots$, $i_{k-1}\sigma = i_k$, $i_k\sigma = i_1$,而$\forall i \in S\setminus \{i_1, i_2, \ldots, i_k\}$, $i\sigma = i$,
     则称$\sigma$为一个\alert{$k$循环置换},记为$(i_1i_2\cdots i_k)$。 $2-$循环置换称为\alert{对换}。
   \end{Def}
 \end{frame}

 \begin{frame}
   \frametitle{6. 置换}
   \begin{Thm}
    每个置换都能被分解成若干个没有共同数字的循环置换的乘积。如果不计这些循环置换的顺序以及略去的$1-$循环置换,这个分解是唯一的。
   \end{Thm}
 \end{frame}

 \begin{frame}
   \frametitle{6. 置换}
   \begin{Thm}
    当$n\geq 2$时,每个$n$次置换都能被分解成若干个对换的乘积。
   \end{Thm}
 \end{frame}

 \begin{frame}
   \frametitle{6. 置换}
   \begin{Thm}
    如果把置换分解成若干个对换的乘积,则对换个数的奇偶性是不变的。
  \end{Thm}
  \pause
  \begin{Def}
    能被分解为偶数个对换的乘积的置换称为\alert{偶置换};能被分解为奇数个对换的乘积的置换称为\alert{奇置换}。
  \end{Def}
  \pause
  \begin{Thm}
    当$n \geq 2$时, $n$次奇置换的个数与$n$次偶置换的个数相等,都等于$\frac{n!}{2}$。
  \end{Thm}
 \end{frame}

 
 
\section{代数运算}
\begin{frame}
  \frametitle{7. 代数运算}
  \begin{minipage}[t]{0.5\linewidth}
  \begin{Thm}
    设$x, y, z \in \mathbb{R}$,则
   \begin{enumerate}
   \item   $x + y = y + x$
   \item   $(x + y) + z = x + (y + z)$
   \item   $0 + x = x + 0 = x$
   \item   $(-x) + x = (-x) + x = 0$
   \item   $x * y = y * x$
   \item   $(x * y) * z = x * (y *z)$
   \item   $1 * x = x * 1 = x$
   \item   $x^{-1} * x = x * x^{-1} = 1$
   \item   $x* (y + z) = x * y + x * z$
   \item   $(y + z) * x = y * x + z * x$
    \end{enumerate}
  \end{Thm}
\end{minipage}\pause
\begin{minipage}[t]{0.5\linewidth}
  \begin{Def}
    设$X$,$Y$,$Z$为任意三个非空集合。一个 从 $X\times Y$ 到 $Z$ 的映射 $\phi$ 称 为 $X$ 与$Y$到$Z$的一个\alert{二元(代数)运算}。当$X=Y=Z$时,则称$\phi$为$X$上的\alert{二元(代数)运算}。
  \end{Def}
\end{minipage}
\end{frame}
\begin{frame}
  \frametitle{7. 代数运算}
  \begin{minipage}[t]{0.49\linewidth}
  \begin{block}{定理7.1}
    设$x, y, z \in \mathbb{R}$,则
   \begin{enumerate}
   \item   $x + y = y + x$
   \item   $(x + y) + z = x + (y + z)$
   \item   $0 + x = x + 0 = x$
   \item   $(-x) + x = (-x) + x = 0$
   \item   $x * y = y * x$
   \item   $(x * y) * z = x * (y *z)$
   \item   $1 * x = x * 1 = x$
   \item   $x^{-1} * x = x * x^{-1} = 1$
   \item   $x* (y + z) = x * y + x * z$
   \item   $(y + z) * x = y * x + z * x$
    \end{enumerate}
  \end{block}\pause
\end{minipage}
\begin{minipage}[t]{0.49\linewidth}
  \begin{Def}
    从集合$X$到$Y$的任一映射称为从$X$到$Y$的\alert{一元(代数)运算}。如果$X=Y$,则从$X$到$X$的映射称为$X$上的\alert{一元(代数)运算}。
  \end{Def}
\end{minipage}
\end{frame}

\begin{frame}
  \frametitle{7. 代数运算}
  \begin{minipage}[t]{0.49\linewidth}
  \begin{block}{定理7.1}
    设$x, y, z \in \mathbb{R}$,则
   \begin{enumerate}
   \item   $x + y = y + x$
   \item   $(x + y) + z = x + (y + z)$
   \item   $0 + x = x + 0 = x$
   \item   $(-x) + x = (-x) + x = 0$
   \item   $x * y = y * x$
   \item   $(x * y) * z = x * (y *z)$
   \item   $1 * x = x * 1 = x$
   \item   $x^{-1} * x = x * x^{-1} = 1$
   \item   $x* (y + z) = x * y + x * z$
   \item   $(y + z) * x = y * x + z * x$
    \end{enumerate}
  \end{block}\pause
\end{minipage}
\begin{minipage}[t]{0.49\linewidth}
  \begin{Def}
    设$A_1, A_2, \cdots, A_n, D$为非空集合。一个从 $A_1\times A_2\times \cdots \times A_n$ 到$D$的映射$\phi$称为 $A_1, A_2, \cdots, A_n$ 到 $D$ 的一个\alert{$n$元(代数)运算}。
    如果$A_1=A_2=\cdots=A_n=D=A$,则称 $\phi$ 为 $A$ 上 的 \alert{$n$元代数运算}。
  \end{Def}
\end{minipage}
\end{frame}

\begin{frame}
  \frametitle{7. 代数运算}
  \begin{minipage}[t]{0.49\linewidth}
  \begin{block}{定理7.1}
    设$x, y, z \in \mathbb{R}$,则
   \begin{enumerate}
   \item   $x + y = y + x$
   \item   $(x + y) + z = x + (y + z)$
   \item   $0 + x = x + 0 = x$
   \item   $(-x) + x = (-x) + x = 0$
   \item   $x * y = y * x$
   \item   $(x * y) * z = x * (y *z)$
   \item   $1 * x = x * 1 = x$
   \item   $x^{-1} * x = x * x^{-1} = 1$
   \item   $x* (y + z) = x * y + x * z$
   \item   $(y + z) * x = y * x + z * x$
    \end{enumerate}
  \end{block}\pause
\end{minipage}
\begin{minipage}[t]{0.49\linewidth}
  \begin{Def}
    设“$\circ$”是集合$X$上的一个二元代数运算。如果$\forall a, b \in X$,恒有\\$a \circ b = b \circ a$, 则称二元代数运算“$\circ$”满足\alert{交换律}。
  \end{Def}
\end{minipage}
\end{frame}

\begin{frame}
  \frametitle{7. 代数运算}
  \begin{minipage}[t]{0.49\linewidth}
  \begin{block}{定理7.1}
    设$x, y, z \in \mathbb{R}$,则
   \begin{enumerate}
   \item   $x + y = y + x$
   \item   $(x + y) + z = x + (y + z)$
   \item   $0 + x = x + 0 = x$
   \item   $(-x) + x = (-x) + x = 0$
   \item   $x * y = y * x$
   \item   $(x * y) * z = x * (y *z)$
   \item   $1 * x = x * 1 = x$
   \item   $x^{-1} * x = x * x^{-1} = 1$
   \item   $x* (y + z) = x * y + x * z$
   \item   $(y + z) * x = y * x + z * x$
    \end{enumerate}
  \end{block}\pause
\end{minipage}
\begin{minipage}[t]{0.49\linewidth}
  \begin{Def}
    设“$\circ$”是集合$X$上的一个二元代数运算。如果$\forall a, b, c \in X$,恒有$(a \circ b) \circ c = a \circ (b \circ c)$, 则称二元代数运算“$\circ$”满足\alert{结合律}。
  \end{Def}
\end{minipage}
\end{frame}

\begin{frame}
  \frametitle{7. 代数运算}
  \begin{minipage}[t]{0.49\linewidth}
  \begin{block}{定理7.1}
    设$x, y, z \in \mathbb{R}$,则
   \begin{enumerate}
   \item   $x + y = y + x$
   \item   $(x + y) + z = x + (y + z)$
   \item   $0 + x = x + 0 = x$
   \item   $(-x) + x = (-x) + x = 0$
   \item   $x * y = y * x$
   \item   $(x * y) * z = x * (y *z)$
   \item   $1 * x = x * 1 = x$
   \item   $x^{-1} * x = x * x^{-1} = 1$
   \item   $x* (y + z) = x * y + x * z$
   \item   $(y + z) * x = y * x + z * x$
    \end{enumerate}
  \end{block}\pause
\end{minipage}
\begin{minipage}[t]{0.49\linewidth}
  \begin{Def}
    设“+”与“$\circ$”是集合$X$上的两个二元代数运算。\\如果$\forall a, b, c \in X$,恒有\[a \circ (b + c) = a \circ b + a \circ c,\] 则称二元代数运算“$\circ$”对“$+$”满足\alert{左分配律}。
    \\如果$\forall a, b, c \in X$,恒有\[(b + c)\circ a = b \circ a + c \circ a,\] 则称二元代数运算“$\circ$”对“$+$”满足\alert{右分配律}。
  \end{Def}
\end{minipage}
\end{frame}

\begin{frame}
  \frametitle{7. 代数运算}
  \begin{minipage}[t]{0.49\linewidth}
  \begin{block}{定理7.1}
    设$x, y, z \in \mathbb{R}$,则
   \begin{enumerate}
   \item   $x + y = y + x$
   \item   $(x + y) + z = x + (y + z)$
   \item   $0 + x = x + 0 = x$
   \item   $(-x) + x = (-x) + x = 0$
   \item   $x * y = y * x$
   \item   $(x * y) * z = x * (y *z)$
   \item   $1 * x = x * 1 = x$
   \item   $x^{-1} * x = x * x^{-1} = 1$
   \item   $x* (y + z) = x * y + x * z$
   \item   $(y + z) * x = y * x + z * x$
    \end{enumerate}
  \end{block}\pause
\end{minipage}
\begin{minipage}[t]{0.49\linewidth}
  \begin{Def}
    设$(X, \circ)$为一个代数系。如果存在一个元素$e\in X$使得对任意的$x\in X$恒有$e\circ x = x \circ e = x$, 则称$e$为“$\circ$”的\alert{单位元素}。
  \end{Def}
\end{minipage}
\end{frame}

\begin{frame}
  \frametitle{7. 代数运算}
  \begin{minipage}[t]{0.49\linewidth}
  \begin{block}{定理7.1}
    设$x, y, z \in \mathbb{R}$,则
   \begin{enumerate}
   \item   $x + y = y + x$
   \item   $(x + y) + z = x + (y + z)$
   \item   $0 + x = x + 0 = x$
   \item   $(-x) + x = (-x) + x = 0$
   \item   $x * y = y * x$
   \item   $(x * y) * z = x * (y *z)$
   \item   $1 * x = x * 1 = x$
   \item   $x^{-1} * x = x * x^{-1} = 1$
   \item   $x* (y + z) = x * y + x * z$
   \item   $(y + z) * x = y * x + z * x$
    \end{enumerate}
  \end{block}\pause
\end{minipage}
\begin{minipage}[t]{0.49\linewidth}
  \begin{Def}
    设$(X, \circ)$为一个代数系,“$\circ$”有单位元素$e$,$a\in X$,如果$\exists b\in X$使得\[a\circ b = b \circ a = e,\]  则称$b$为$a$的\alert{逆元素}。
  \end{Def}
\end{minipage}
\end{frame}

\begin{frame}
  \frametitle{7. 代数运算}
  \begin{Def}
    设$(S,+)$与$(T, \oplus)$为两个代数系。如果存在一个一一对应$\phi:S\to T$, 使得$\forall x, y \in S$,有
    \begin{align*}
      \phi(x+y) &= \phi(x) \oplus \phi(y),
    \end{align*}
    则称代数系$(S,+)$与$(T, \oplus)$\alert{同构},并记为$S\cong T$, $\phi$称为这两个代数系之间的一个同构。
  \end{Def}
\end{frame}

\begin{frame}
  \frametitle{7. 代数运算}
  \begin{Def}
    设$(S,+, \circ)$与$(T, \oplus, *)$为两个代数系。如果存在一个一一对应$\phi:S\to T$, 使得$\forall x, y \in S$,有
    \begin{align*}
      \phi(x+y) &= \phi(x) \oplus \phi(y),\\
      \phi(x\circ y)&= \phi(x) * \phi(y),
    \end{align*}
    则称代数系$(S,+,\circ)$与$(T, \oplus, *)$\alert{同构},并记为$S\cong T$, $\phi$称为这两个代数系之间的一个同构。
  \end{Def}
\end{frame}

\begin{frame}
  \frametitle{7. 代数运算}
  \begin{tabular}{cc|c}
    p& q& p $\land$ q\\
    \hline
    T&T&T\\
    T&F&F\\
    F&T&F\\
    F&F&F\\
  \end{tabular}\hspace{1cm}
  \begin{tabular}{cc|c}
    p& q& p $\lor$ q\\
    \hline
    T&T&T\\
    T&F&T\\
    F&T&T\\
    F&F&F\\
  \end{tabular}\hspace{1cm}
  \begin{tabular}{c|c}
    p& $\lnot$ p\\
    \hline
    T&F\\
    F&T\\
  \end{tabular}

  \vspace{1cm}\pause
    \begin{tabular}{cc|c}
    x& y& x $\land$ y\\
    \hline
    1&1&1\\
    1&0&0\\
    0&1&0\\
    0&0&0\\
  \end{tabular}\hspace{1cm}
  \begin{tabular}{cc|c}
    x& y& x $\lor$ y\\
    \hline
    1&1&1\\
    1&0&1\\
    0&1&1\\
    0&0&0\\
  \end{tabular}\hspace{1cm}
  \begin{tabular}{c|c}
    x& $\bar{x}$\\
    \hline
    1&0\\
    0&1\\
  \end{tabular}

\end{frame}

\section{集合的特征函数}
\begin{frame}
  \frametitle{8. 集合的特征函数}
  \begin{Def}
    设$X$是一个集合,$E \subseteq X$。 $E$的\alert{特征函数}$\chi_E:X\to \{0,1\}$定义为
    \begin{equation*}
      \chi_E(x)=
      \begin{cases}
        1 & \text{如果} x \in E,\\
        0 & \text{如果} x \notin E.
      \end{cases}
    \end{equation*}
  \end{Def}
\end{frame}
\begin{frame}
  \frametitle{8. 集合的特征函数}
  \begin{Def}
    令$Ch(X) = \{\chi |\chi:X \to \{0,1\}\}$。
    $\forall \chi, \chi' \in Ch(X)$及$x \in X$,
    \begin{align}
      (\chi \lor \chi')(x) &= \chi(x) \lor \chi'(x)\nonumber\\
      (\chi \land \chi')(x) &= \chi(x) \land \chi'(x)\nonumber\\
      \bar{\chi}(x) &=   \overline{\chi(x)}
    \end{align}
  \end{Def}
  \begin{Thm}
    设$X$是一个集合,则代数系$(2^X, \cup, \cap, ^c)$与$(Ch(X), \lor, \land, \bar{} \ )$同构。
  \end{Thm}
\end{frame}

\begin{frame}
  \frametitle{习题}
    \begin{Exercise}
  设$X=\{a,b,c\}, Y=\{0,1\}, Z=\{2,3\}$。$f:X \to Y, f(a) = f(b) = 0, f(c) = 1$;
  $g:Y\to Z, g(0) = 2, g(1) = 3$。试求$g\circ f$。
  \end{Exercise}
  \begin{Exercise}
    设$f:X \to Y$,$C \subseteq Y$,$D \subseteq Y$,证明

    $f^{-1}(C \setminus D) = f^{-1}(C) \setminus f^{-1}(D)$
  \end{Exercise}
    \begin{Exercise}
    设$f:X \to Y$,$A \subseteq X$,$B \subseteq X$,证明

    $f(A \setminus B) \supseteq f(A) \setminus f(B)$
    
  \end{Exercise}
  \begin{Exercise}
    设$f:X\to Y$,$A \subseteq X$,则$(f(A))^c \subseteq f(A^c)$成立吗?$ f(A^c)\subseteq (f(A))^c$成立吗?
  \end{Exercise}

\end{frame}

\begin{frame}
  \frametitle{习题}
  \begin{Exercise}
    设$f:X\to Y$, 证明:$f$为满射当且仅当$\forall E \in 2^Y, f(f^{-1}(E)) = E$。
  \end{Exercise}

  \begin{Exercise}
    设$f:X\to Y$, 证明:$f$为单射当且仅当$\forall F \in 2^X, f^{-1}(f(F)) = F$。    
  \end{Exercise}
    \begin{Exercise}
    设$f:X \to Y$,$g:Y \to Z$,$A \subseteq Z$,证明:$(gf)^{-1}(A) = f^{-1}(g^{-1}(A))$。
  \end{Exercise}
  \begin{Exercise}
    设$N=\{1,2,\ldots\}$,试构造两个映射$f:N \to N$与$g:N\to N$,使得$fg = I_N$,
    但$gf \neq I_N$。
  \end{Exercise}
\end{frame}
\begin{frame}
  \frametitle{习题}
 \begin{Exercise}
    设$f:X \to Y$,

    (1)如果存在唯一的一个映射$g:Y\to X$,使得$gf = I_X$,那么$f$是否可逆呢?

    (2)如果存在唯一的一个映射$g:Y\to X$,使得$fg = I_Y$,那么$f$是否可逆呢?

  \end{Exercise}
  \begin{Exercise}
    是否存在一个从集合$X$到$X$的一一对应,使得$f=f^{-1}$,但$f \neq I_X$?
  \end{Exercise}
\end{frame}

\end{CJK*}
\end{document}

%%% Local Variables:
%%% mode: latex
%%% TeX-master: t
%%% End:
